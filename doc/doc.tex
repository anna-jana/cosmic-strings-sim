\documentclass[a4paper]{article}

\usepackage[utf8]{inputenc}
\usepackage{latexsym,amssymb,amsmath,amstext}
\usepackage{graphicx}
\usepackage[backend=bibtex, sorting=none]{biblatex}
\usepackage{subfigure}
\usepackage{float}

\bibliography{refs}

\begin{document}

\section{Background Cosmology}
Using the metric
\begin{equation}
    \mathrm{d}s^2 = -\mathrm{d}t^2 + a(t) \mathrm{d} x_i \mathrm{d} x_i
\end{equation}
and the Hubble parameter in the radiation domainted era ($\rho \sim a^{-4}$)
\begin{equation}
    H = \frac{1}{2t}
\end{equation}
$a$ is the scale factor obeying the Friedmann equation
\begin{equation}
    2 M_\mathrm{pl}^2 H^2 = \rho_\mathrm{total} = \frac{\pi^2}{30} g_* T^4
\end{equation}

Define the conformal time $\tau$ such that:
\begin{equation}
    \frac{\mathrm{d} \tau}{\mathrm{d} t} = \frac{1}{a(t)}
\end{equation}
hence:
\begin{equation}
    \frac{\mathrm{d}}{\mathrm{d} t} = \frac{\mathrm{d \tau}}{\mathrm{d} t} \frac{\mathrm{d}}{\mathrm{d} \tau} = \frac{1}{a(t)} \frac{\mathrm{d}}{\mathrm{d} \tau}
\end{equation}

\begin{align}
    \frac{\mathrm{d}^2}{\mathrm{d} t^2} = \frac{\mathrm{d}}{\mathrm{d} t} \frac{1}{a} \frac{\mathrm{d}}{\mathrm{d} \tau}
    = - \frac{\dot{a}}{a^2} \frac{\mathrm{d}}{\mathrm{d} \tau} + \frac{1}{a} \frac{\mathrm{d}}{\mathrm{d} t} \frac{\mathrm{d}}{\mathrm{d} \tau}
     = - \frac{\dot{a}}{a^2} \frac{\mathrm{d}}{\mathrm{d} \tau} + \frac{1}{a^2} \frac{\mathrm{d}^2}{\mathrm{d} \tau^2}
\end{align}

where
\begin{equation}
    \frac{\mathrm{d} 1/a}{\mathrm{d} \tau} = - a' / a^2 = - a \dot{a} / a^2 = - H
\end{equation}

\begin{equation}
    a = a_1 t^{1/2}
\end{equation}
\begin{equation}
    H = \frac{1}{2t}
\end{equation}

\begin{equation}
    \tau(t) = \int_0^t \mathrm{d} t' a^{-1}(t') = \int_0^t \mathrm{d} t' a_1 t^{-1/2} = 2 a_1 t^{1/2}
\end{equation}

\section{Model}

Based on the paper by Gorghetto et al. \cite{axions_from_strings}.
We use the string identification method by \cite{axion_dark_matter_strings_and_their_cores}.
The article \cite{improved_estimation_hiramatsu} is also important.

\begin{equation}
    \mathcal{L} = - \frac{1}{2} ( \partial \phi )^2 + V(\phi)
\end{equation}

\begin{equation}
    V(\phi) = \frac{m_r^2}{2 f_a^2}\left( |\phi|^2 - \frac{f_a^2}{2} \right)^2
\end{equation}

\begin{equation}
    \varrho = T_{00} = \underbrace{\frac{1}{2} \dot{a}^2 + \frac{1}{2} | \nabla a|^2}_{= \varrho_\mathrm{axion}}
    + \underbrace{ \frac{1}{2} \dot{r}^2 + \frac{1}{2} | \nabla r|^2 + V(r) }_{= \varrho_\mathrm{radion}}
    + \left(\frac{r^2}{2 f_a^2} + \frac{r}{f_a} \right) (\dot{a}^2 + |\nabla a|^2)
\end{equation}

Equation of Motion from the Lagragian:
\begin{equation}
    \ddot{\phi} + 3 H \dot{\phi} - \nabla^2 \phi + \phi \frac{m_r^2}{f_a^2} \left( |\phi|^2 - \frac{f_a^2}{2} \right) = 0
\end{equation}

Here the nabla is with respect to physical coordinates. They are related to comoving  coordinates by $x_\mathrm{physical} = a(t) x_\mathrm{comoving}$.
In comoving coordinates in the equation of motion the nabal is to be replaced $\nabla \to \nabla / a(t)^2$.
\begin{equation}
    (\nabla f)_i = \frac{\partial f}{\partial x_i} = \frac{\partial f}{\partial a(t) x_{\mathrm{comoving}, i}} = (\tilde{\nabla} f)_i / a^2(t)
\end{equation}

% \begin{equation}
%     \ddot{\phi} + 3 H \dot{\phi} - \nabla^2 \phi + \phi m_r^2 \left( \frac{|\phi|^2}{f_a^2} - \frac{1}{2} \right) = 0
% \end{equation}

Divide by $m_r^2$ and $f_a$:
\begin{equation}
    \frac{\mathrm{d}^2 {\phi}}{\mathrm{d} t^2 f_a m_r^2} + 3 \frac{H}{m_r} \frac{\mathrm{d} \phi}{\mathrm{d} t m_r f_a} - \frac{\nabla^2 \phi}{m_r^2 f_a} + \frac{\phi}{f_a} \left( \frac{ |\phi|^2 } { f_a^2 } - \frac{1}{2} \right) = 0
\end{equation}

Using the replacements:
\begin{align}
    t' &= m_r t \\
    H' &= H / m_r \\
    x' &= m_r x \\
    \phi' &= \phi / f_a
\end{align}

we find:
\begin{equation}
    \frac{\mathrm{d}^2 \phi'}{\mathrm{d} t'^2} + 3 H' \frac{\mathrm{d} \phi'}{\mathrm{d} t'} - \nabla'^2 \phi + \phi' \left( |\phi'|^2 - \frac{1}{2} \right) = 0
\end{equation}

We rewrite the time derivative as a derivative with respect to $\tau$.
Now use unprimed quantities and use the prime as the derivative with respect to $\tau$.
\begin{equation}
    - 1/a H \phi' + 1/a^2 \phi'' + 3 H / a \phi'
    - \nabla^2 \phi + \phi \left( |\phi|^2 - \frac{1}{2} \right) = 0
\end{equation}

\begin{equation}
    \frac{\mathrm{d} a(t)}{\mathrm{d} \tau} = \frac{\mathrm{d} t}{\mathrm{d} \tau} \dot{a} = \dot{a} / \frac{\mathrm{d} \tau}{\mathrm{d} t} = \dot{a} / (1 / a(t)) = \dot{a} a
\end{equation}

\begin{equation}
    \phi = \psi / a
\end{equation}

\begin{equation}
    \frac{\mathrm{d}}{\mathrm{d} \tau} ( \psi / a(t) ) = \frac{\mathrm{d} \psi}{\mathrm{d} \tau} / a(t) - \frac{\mathrm{d} a(t)}{\mathrm{d} \tau} \psi / a^2(t)
    = \frac{\mathrm{d} \psi}{\mathrm{d} \tau} / a(t) - \dot{a} a \psi / a^2(t)
    = \psi' / a - \psi H
\end{equation}

\begin{equation}
    \frac{d H}{d \tau} = \frac{d t}{d \tau} \frac{d H}{d t} = a (\ddot{a} / a - \dot{a}^2 / a^2) = \ddot{a} - a H^2
\end{equation}

\begin{equation}
    \frac{d^2}{d \tau^2} (\psi / a) = \frac{d}{d \tau} (\psi' / a - \psi H)
    = \psi'' / a - \psi' H - (\psi' H + \psi \frac{d H}{d \tau})
    = \psi'' / a - 2 \psi' H - \psi ( \ddot{a} - a H^2 )
\end{equation}

\begin{equation}
    1/a^2 ( \psi'' / a - 2 \psi' H - \psi ( \ddot{a} - a H^2 ) ) + 2 H / a (\psi' / a - \psi H)
    - \nabla^2 \psi / a + \psi / a \left( |\psi / a|^2 - \frac{1}{2} \right) = 0
\end{equation}

\begin{equation}
    \psi'' / a^3 - 2 \psi' H / a^2 - \psi \ddot{a} / a^2 + \psi H^2 / a
    + 2 \psi' H / a^2 - 2 H^2 \psi / a
    - \nabla^2 \psi / a + \psi / a^3 ( |\psi|^2 - \frac{a^2}{2} ) = 0
\end{equation}

\begin{equation}
    \psi'' / a^3 - \psi \ddot{a} / a^2 - H^2 \psi / a
    - \nabla^2 \psi / a + \psi / a^3 ( |\psi|^2 - \frac{a^2}{2} ) = 0
\end{equation}

\begin{equation}
    \psi'' + \psi (\underbrace{- \ddot{a} a - H^2 a^2}_{= 0})
    - \nabla^2 \psi a^2 + \psi ( |\psi|^2 - \frac{a^2}{2} ) = 0
\end{equation}

Finally we go to comoving coordinates $\tilde{x}$ defined by
\begin{equation}
    \tilde{x} = x' / a
\end{equation}
and denote the spacial derivative with respect to the comoving coordinates by $\tilde{\nabla}$.

\begin{equation}
    \psi'' - \tilde{\nabla}^2 \psi + \psi ( |\psi|^2 - \frac{a^2}{2} ) = 0
\end{equation}

\begin{equation}
    \phi_\mathrm{physical} = f_a \psi / a(t)
\end{equation}

\section{Energies}

\subsection{Decomposition into Axion and Radial Mode}
\begin{equation}
    \phi(x) = \frac{r(x) + f_a}{\sqrt{2}} e^{i \varphi(x) / f_a }
\end{equation}

\subsection{The Hamiltonian Density}
In physical coordinates and field basis:
\begin{equation}
    \mathcal{H} =
    \underbrace{\frac{1}{2} \dot{\varphi}^2 + \frac{1}{2} (\nabla \varphi)^2}_{= \mathrm{axion}} +
    \underbrace{\frac{1}{2} \dot{r}^2 + \frac{1}{2} (\nabla r)^2 + V(r)}_{= \mathrm{radial}} +
    \underbrace{\left( \frac{r^2}{f_a^2} + 2 \frac{r}{f_a} \right) \left( \frac{1}{2} \dot{\varphi}^2 + \frac{1}{2} (\nabla \varphi)^2 \right)}_{= \mathrm{interaction}}
\end{equation}

\subsection{Axion:}

field definition and conversion:
\begin{equation}
    \varphi = f_a \theta = f_a \arg(\phi_\mathrm{physical}) = f_a \arg(\psi)
\end{equation}

kinetic:
\begin{align}
    &\frac{1}{2} \dot{\varphi}^2 = \frac{1}{2} f_a^2 \dot{\theta}^2
    = \frac{1}{2} f_a^2 \left( \frac{\mathrm{d} \theta}{\mathrm{d} t} \right)^2
    = \frac{1}{2} f_a^2 \left( \frac{\mathrm{d} \theta}{\mathrm{d} t' / m_r} \right)^2 \\
    &= \frac{1}{2} f_a^2 m_r^2 \left( \frac{\mathrm{d} \tau}{\mathrm{d} t'} \frac{\mathrm{d} \theta}{\mathrm{d} \tau} \right)^2
    = \frac{1}{2} f_a^2 m_r^2 \left( \frac{1}{a(t)} \frac{\mathrm{d} \theta}{\mathrm{d} \tau} \right)^2
    = \frac{1}{2} \left( \frac{f_a m_r}{a(t)} \right)^2 \left( \partial_\tau \theta \right)^2
\end{align}

\begin{align}
    \partial_\tau \theta &= \partial_\tau \arg(\phi) = \partial_\tau \arg(\psi) = \partial_\tau \arctan \left( \frac{ \Im \psi }{ \Re \psi } \right)
    = \frac{ \frac{ \partial_\tau \Im \psi }{ \Re \psi } - \frac{ \Im \psi \cdot \partial_\tau \Re \psi}{(\Re \psi)^2}}{ 1 + \left( \frac{\Im \psi}{\Re \psi} \right)^2} \\
                         &= \frac{ \Re \psi \partial_\tau \Im \psi - \Im \psi \cdot \partial_\tau \Re \psi}{ (\Re \psi)^2 + (\Im \psi)^2} =  \frac{ \Re \psi \partial_\tau \Im \psi - \Im \psi \cdot \partial_\tau \Re \psi}{|\psi|^2}
\end{align}

gradient:
\begin{align}
 &\frac{1}{2} (\nabla \varphi)^2 =
 \frac{1}{2} f_a^2 \sum_i \left(\frac{\partial \theta}{\partial x_i}\right)^2 = \\
 &\frac{1}{2} f_a^2 \sum_i \left(\frac{\partial \theta}{\partial x'_i / m_r}\right)^2 =
 \frac{1}{2} f_a^2 m_r^2 \sum_i \left(\frac{\partial \theta}{\partial x'_i}\right)^2 =
 \frac{1}{2} f_a^2 m_r^2 (\nabla' \theta)^2
\end{align}

\begin{equation}
    \frac{1}{2} (\nabla \varphi)^2 = \frac{1}{2} f_a^2 m_r^2 (\tilde{\nabla} \varphi)^2 / a(t)^2
\end{equation}

\subsection{Radial Mode:}

Field definition and conversion
\begin{equation}
    \phi(x) = \frac{r(x) + f_a}{\sqrt{2}} e^{i \theta(x) }
\end{equation}

\begin{equation}
    |\phi| = \frac{r + f_a}{\sqrt{2}}
\end{equation}

\begin{equation}
    r = \sqrt{2} |\phi| - f_a
    = \sqrt{2} f_a |\psi| / a(t) - f_a
    = f_a \left( \sqrt{2} |\psi| / a(t) - 1 \right)
    \equiv f_a r'
\end{equation}

\begin{equation}
    r' = r / f_a = \sqrt{2} |\psi| / a(t) - 1
\end{equation}

kinetic:
\begin{equation}
    \frac{1}{2} \dot{r}^2 = \frac{1}{2} (\frac{\partial}{\partial t} r)^2
    = \frac{1}{2} (\frac{f_a m_r}{a(t)})^2 \left( \partial_\tau r' \right)^2
\end{equation}

\begin{align}
    \partial_\tau r' = \partial_\tau \left( \sqrt{2} |\psi| / a - 1 \right)
    &= \sqrt{2} \partial_\tau ( |\psi| / a ) \\
    &= \sqrt{2} (\partial_\tau |\psi| / a - \partial_\tau a |\psi| / a^2) \\
    &= \sqrt{2} (\partial_\tau |\psi| / a - \dot{a} a |\psi| / a^2) \\
    &= \sqrt{2} (\partial_\tau |\psi| / a - H |\psi|)
\end{align}

\begin{align}
    \partial_\tau |\psi| &= \partial_\tau \sqrt{ (\Re \psi)^2 + (\Im \psi)^2 } \\
                         &= \frac{ \partial_\tau ((\Re \psi)^2 + (\Im \psi)^2) }{ 2 \sqrt{ (\Re \psi)^2 + (\Im \psi)^2 } } \\
                         &= \frac{ 2 \Re \psi \partial_\tau \Re \psi + 2 \Im \psi \partial_\tau \Im \psi }{ 2 \sqrt{ (\Re \psi)^2 + (\Im \psi)^2 } } \\
                         &= \frac{ \Re \psi \partial_\tau \Re \psi + \Im \psi \partial_\tau \Im \psi }{ |\psi| }
\end{align}

gradient:
\begin{equation}
    \frac{1}{2} (\nabla r)^2 = \frac{1}{2} f_a^2 m_r^2 ( \nabla' r')^2
    = \frac{1}{2} f_a^2 m_r^2 ( \tilde{\nabla} r')^2 / a(t)^2
\end{equation}

potential:
\begin{equation}
    V(\phi) = \frac{m_r^2}{2 f_a^2}\left( |\phi|^2 - \frac{f_a^2}{2} \right)^2
\end{equation}

\begin{align}
    &V(\phi) = V(|\phi|) = V(\frac{r + f_a}{\sqrt{2}})
    = V \left(f_a \left( \frac{r' + 1}{\sqrt{2}} \right) \right) \\
    &= \frac{m_r^2}{2 f_a^2}\left( \left(f_a \left( \frac{r' + 1}{\sqrt{2}} \right) \right)^2 - \frac{f_a^2}{2} \right)^2 \\
    &= m_r^2 f_a^2 / 8 ( (r' + 1)^2 - 1)
    = m_r^2 f_a^2 / 8 ( r'^2 + 2 r' )^2
\end{align}

\subsection{Interaction:}
\begin{equation}
\left( \frac{r^2}{f_a^2} + 2 \frac{r}{f_a} \right) \left( \frac{1}{2} \dot{\varphi}^2 + \frac{1}{2} (\nabla \varphi)^2 \right)
= f_a^2 m_r^2 (r'^2 + 2 r') \left( \frac{1}{2} (\partial_\tau \theta )^2 + \frac{1}{2} (\tilde{\nabla} \theta)^2 / a(t)^2 \right)
\end{equation}

\newpage
\section{Spectrum Estimation}

We use the method described in \cite[sec. 3.2]{improved_estimation_hiramatsu}.

\begin{equation}
    \frac{1}{2} \langle \dot{a}(\vec{k}, t)^* \dot{a}(\vec{k}', t) \rangle = \frac{(2\pi)^3}{k^2} \delta^{(3)}(\vec{k} - \vec{k}') P(k, t)
\end{equation}

\begin{equation}
    \tilde{\dot{a}}(\vec{x}_\mathrm{phys}) \equiv W(\vec{x}_\mathrm{phys}) \dot{a}(\vec{x}_\mathrm{phys})
\end{equation}

\begin{equation}
    \tilde{P}(k) \equiv \frac{k^2}{V} \int \frac{d \Omega_k}{4\pi} \frac{1}{2} \left| \tilde{\dot{a}}( \vec{k} ) \right|^2
\end{equation}

\begin{equation}
    \hat{P}(k) \equiv \frac{k^2}{V} \int \frac{d k'}{2 \pi^2} M^{-1}(k, k') \tilde{P}(k')
\end{equation}

\begin{equation}
    \int \frac{k'^2 d k'}{2 \pi^2} M^{-1}(k, k') M(k', k'') = \frac{2\pi^2}{k^2} \delta(k - k'')
\end{equation}

\begin{equation}
    M(k, k') \equiv \frac{1}{V^2} \int \frac{d \Omega_k}{4 \pi} \frac{d \Omega_{k'}}{4 \pi} \left| W(\vec{k} - \vec{k'}) \right|^2
\end{equation}

\begin{enumerate}
    \item Detect Strings
    \item Compute $W(\vec{x})$
    \item Compute $\dot{a}(\vec{x})$
    \item Compute $\tilde{\dot{a}}(\vec{x}) = a(\vec{x}) \cdot W(\vec{x})$
    \item Compute $\tilde{P}(k)$ from $\tilde{\dot{a}}(\vec{x})$
    \item Compute FFT of $W(\vec{x})$
    \item Compute $M(k, k')$ from $W(\vec{k})$
    \item Invert $M(k, k')$
    \item Compute $\hat{P}(k)$ using $M^{-1}(k, k')$ and $\tilde{P}(k)$
\end{enumerate}

\subsection{Instantaneous Emission Spectrum}
We define the instantaneous emission spectrum into axions as
\begin{equation}
    F[x = k / H, y = m_r / H] = A / a^3 \frac{\partial}{\partial t} \left( a^3 \frac{\partial \rho_a}{\partial k} \right)
\end{equation}

\newpage
\section{Strings}

\subsection{String Identification}
A plaquette in the grid contains a string iff the complex field $\phi$ wraps the fundamental domain once around the plaquette.
This is equivalent to if it crosses the real axis with the same handedness twice.
The real axis lies between $\phi_1$ and $\phi_2$ iff
\begin{equation}
	r(\phi_1, \phi_2) = \mathrm{Im}(\phi_1) \mathrm{Im}(\phi_2) < 0.
\end{equation} 
The handedness of a rotation between $\phi_1$ and $\phi_2$ is given by
\begin{equation}
	h(\phi_1, \phi_2) = \mathrm{sgn} \, \mathrm{Im} (\phi_1 \phi_2^*).
\end{equation}
Hence the effective number of crossings is given by
\begin{equation}
	c(\phi_1, \phi_2) = r(\phi_1, \phi_2) h(\phi_1, \phi_2)
\end{equation}
The effective number of crossings of the four sides at the courners of the plaquett is
\begin{equation}
	c(\phi_1, \phi_2) + c(\phi_2, \phi_3) + c(\phi_3, \phi_4) + c(\phi_4, \phi_1).
\end{equation}
This value has to be exactly 2 for the plaquett to constain a string core.

\subsection{String Length Parameter}
We define the string length per Hubble volume as
\begin{equation}
    \xi \equiv \lim_{V \to \infty} \frac{l}{vol(V)} t^2
\end{equation}
where $l$ is the total length of strings within $V$.

After identifiying strings we sum the distance between neighboring points on each string. Hence the string length parameter is given as 
\begin{equation}
	\xi = \frac{a(t) l \Delta x}{a^3(t) L^3} t^2
\end{equation}

\subsection{String Velocity}
We use the algorithm of \cite[appendix A.2, eq. A.10]{axion_dark_matter_strings_and_their_cores}.
It is given as
\begin{equation}
	\gamma^2 v^2 = \frac{\dot{\phi} \dot{\phi}'}{c^2} \left(1 + \frac{|\phi|^2}{8 c^2} \right) + \frac{\phi' \dot{\phi} + \phi \dot{\phi}')^2}{16 c^4}
\end{equation}

\newpage 
\section{The String Solution}

The Ansatz for a straight string is
\begin{equation}
	\psi = f(r) e^{\alpha}
\end{equation}
where we work in cylindrical coordinates (the string is along the z-axis, $\alpha$ is the angle around the z-axis and $r$ is the distance to the z-axis).
Here $f$ has to solve the time-independent equation of motion
????

\begin{equation}
	\mu = \pi f_a^2 \log \frac{m_r \gamma}{H \sqrt{\xi}}
\end{equation}

\newpage
\section{Number Density Extrapolation}

The evolution of the energy density of strings $\rho_\mathrm{strings}$
and of free axions $\rho_\mathrm{axions}$ is given by
\begin{equation}
	\label{eq:string_energy_eom}
	\dot{\rho}_\mathrm{strings} + 2 H \rho_\mathrm{strings} = - \Gamma
\end{equation}
and
\begin{equation}
	\label{eq:axion_energy_eom}
	\dot{\rho}_\mathrm{axions} + 4 H \rho_\mathrm{axions} = + \Gamma
\end{equation}
respectively.
Here $\Gamma$ is the rate of emission of axions from the string network.
Here we made the approximation that no energy is released into the heavy radial modes. 

The energy density of strings can be approximately computed from their length and energy per unit length as
\begin{equation}
	\rho_\mathrm{strings} = \mu \times l = \pi f_a^2 \log \frac{m_r \gamma}{H \sqrt{\xi}} \frac{\xi}{t^2}
\end{equation}

This allows us to compute the rate $\Gamma$ as 
\begin{equation}
	\Gamma = 2 \pi f_a^2 \frac{\xi}{t^3} \left( \log \frac{m_r \gamma}{H \sqrt{\xi}} - 1  \right)
\end{equation}

Now we want to find the number density of axions.
\begin{equation}
	\label{eq:number_density_from_spectrum}
	n_\mathrm{axion}(t) = \int \mathrm{d} k \frac{P(k)}{k} 
\end{equation}
where $k$ is the physical momentum which is related to the comoving as $k_\mathrm{physical} = k_\mathrm{comoving} / a(t)$.
We want to relate this to the rate $\Gamma$ and a shape of the spectrum $P(k)$.
For later reference we also express the axion energy density in term of the spectrum $P(k)$
\begin{equation}
	\rho_\mathrm{axion}  = \int \mathrm{d} k P(k)
\end{equation}
DANGER: there are potentially factors of $2\pi$ or so from the definition of $P(k)$ somewhere 
DANGER: gorghetto has physical momenta and hirmatsu/kawasaki comoving!!!!

\subsection{The Instantaneous Emission Spectrum (Gorghetto)}
First lets use the Ansatz by Gorghetto. 
We change the presentation of the formalism by Gorghetto (the following formula is a definition for us but is a consequence in the Gorghetto paper).
They define the instantaneous emission spectrum $F[x,y]$ as
\begin{equation}
	\label{eq:F_def}
	F[x = k' / H = a/a' k / H, y = m_r / H] = \frac{H}{\Gamma} \frac{1}{a^3} \frac{\partial}{\partial t} \left( a^3 \frac{\partial \rho_a}{\partial k a / a'} \right)
\end{equation}
$F$ is normalized to 1:
\begin{align}
	\int \mathrm{d} k F[a/a'k/H, m_r/H] &= \frac{H}{\Gamma} \frac{1}{a^3} \frac{1}{H} \frac{\partial}{\partial t} a^3 \int \mathrm{d} k \frac{\partial \rho_a}{\partial k a / a'} \\
	&= \frac{1}{\Gamma} \frac{1}{a^4} \frac{\partial}{\partial t} a'^4 \rho_a \\
	&= \frac{1}{\Gamma} \frac{1}{a^4} (3 \dot{a} a^2 \rho_a + a^3 \dot{\rho_a}) \\
	&= \frac{1}{\Gamma} \frac{1}{a^4} a^4 (3 H \rho_a + \dot{\rho_a}) \\
	&= \frac{1}{\Gamma} (4 H \rho_a - 4 H \rho_a + \Gamma) \\
	&= \frac{1}{\Gamma} \Gamma \\
	&= 1
\end{align}
This is a bit weird with the $a$ and $a'$ buissness.
Technically does $k$ depend on $t$ so we can't switch the time-derivative and the k-integral.
Maybe we could write this more clearly.

We can straight forwardly integrate the spectrum using eq. \eqref{eq:number_density_from_spectrum}  and use the expression in eq. \eqref{eq:F_def} to find the number density in terms of the instantaneous emission spectrum:
\begin{align}
	\label{eq:number_density_from_F}
	n_a(t) &= \int \frac{\mathrm{d} k}{k} \frac{\partial \rho_a}{\partial k} \\
	&= \int \mathrm{d} t' \frac{\Gamma(t')}{H(t')} \left(\frac{R(t')}{R(t)}\right)^3 \int \frac{\mathrm{d} k}{k} F\left[\frac{kR(t)}{H(t')R(t')}, \frac{m_r}{H}\right]
\end{align}

\subsection{Comparision with Hiramatsu and Kawasaki Formalism}
Now we compare this to the formalism by the Hiramatsu group:
The energy in a comoving volume is
\begin{equation}
	E(t) = a(t)^4 \rho_\mathrm{axion}(t)
\end{equation}
The number density in a comoving volume is
\begin{equation}
	N(t) = a(t)^3 n_\mathrm{axion}.
\end{equation}

First of all their defintions lead to the same result:
\begin{align}
	\label{eq:hiramatsu_is_the_same_as_gorghetto}
	R(t)^3 n_a(t) = N(t) &= \int \partial_t' N(t) = \int \partial_t' \left( R(t')^3 \int \frac{\mathrm{d} k}{k} \frac{\partial \rho_a}{k} \right) \\
	&= \int \mathrm{d} t' \int \frac{\mathrm{d} k}{k} \frac{\partial \rho_a}{k} \\
	&= \int \mathrm{d} t' \frac{\Gamma(t')}{H(t')} R(t')^3 \int \frac{\mathrm{d} k}{k} F\left[\frac{ka(t)}{H(t')a(t')}, \frac{m_r}{H}\right]
\end{align}

Now we use their formalism.
Using the definition of $E(t)$ we can rewrite the eom for the axion energy density eq. \eqref{eq:axion_energy_eom} as 
\begin{equation}
	\dot{E}(t) = R^4(t) \Gamma(t)
\end{equation}

In the Kawasaki paper \cite{improved_estimation_hiramatsu} they use 
\begin{equation}
	\frac{\frac{\mathrm{d} N}{\mathrm{d} t}}{\frac{\mathrm{d} E}{\mathrm{d} t}} =: \overline{k^{-1}}(t)
\end{equation}
as the characteristic parameter of the spectrum.
For this they write 
\begin{align}
	n_a(t) R(t)^3 &= N(t) = \int \mathrm{d} t' \dot{N} = \int \mathrm{d} t' \dot{E} \frac{\dot{N}}{\dot{E}} \\
	&= \int \mathrm{d} t' R^4(t) \Gamma(t) \frac{\dot{N}}{\dot{E}}
\end{align}
By comparing this equation with eq. \eqref{eq:hiramatsu_is_the_same_as_gorghetto} we find:
\begin{equation}
	R(t) \overline{k^{-1}}(t) = \int \frac{\mathrm{d} k}{k} \frac{F[k/H, m_r/H]}{H}
\end{equation}

\subsection{Computing the Number Density}
We can now compute the number density using eq. \eqref{eq:number_density_from_F}.
This is useful because we can find an Ansatz for the instantaneous emission spectrum $F[x, y]$.
We assume that this is a single power law $q$ with cutoffs in the UV at $x = x_0$ and in the IR at the Hubble scale $y = m_r/H$
\begin{equation}
	F[x, y] \sim \left(\frac{x_0}{x}\right)^q \Theta(x - x_0) \Theta(y - x)
\end{equation}
We know already that $F[x,y]$ is normalized in $x$ to 1. Let us compute the normalization
\begin{align}
	\int_{x_0}^{y} \mathrm{d} x \left(\frac{x_0}{x}\right)^q &= x_0^q \frac{x^{-q + 1}}{-q + 1}\Big|_{x_0}^y \\
	&= \frac{x_0^q}{1 - q}(y^{1 - q} - x_0^{1 - q}) = \frac{x_0}{q - 1} (1 - (x_0/y)^{q - 1})
\end{align}
Hence we find our normalized Ansatz for the spectrum
\begin{equation}
	F[x, y] = \frac{1 - q}{x_0(1 - (x_0/y)^{q - 1})} \left(\frac{x_0}{x}\right)^q \Theta(x - x_0). \Theta(y - x)
\end{equation}

We start by computing the momentum integral 
\begin{align}
	\int \frac{\mathrm{d} x}{x} \frac{F[x, y]}{H} &= 
	\int_{x_0}^{m_r/H} \frac{\mathrm{d} x}{x} \frac{q - 1}{1 - \left(\frac{x_0}{m_r/H}\right)^{q - 1}} \frac{1}{x_0} \left( \frac{x_0}{x} \right)^q \frac{1}{H} \\
	&=   \frac{q - 1}{1 - \left(\frac{x_0}{m_r/H}\right)^{q - 1}} \frac{1}{x_0 H} x_o^q \int_{x_0}^{m_r/H} \mathrm{d} x x^{-q - 1} \\
	&= \frac{q - 1}{1 - \left(\frac{x_0}{m_r/H}\right)^{q - 1}} \frac{1}{x_0 H} x_o^q \frac{1}{q} \left( x_0^{-q} - \left( \frac{H}{m_r}\right)^q \right) \\
	&= \frac{1 - q^{-1}}{m_r} \frac{\left(\frac{m_r}{H x_0}\right)}{1 - \left(\frac{m_r}{H x_0}\right)^{1 - q}} \left( 1 - \left(\frac{m_r}{H x_0}\right)^{-q} \right) \\
	&= \frac{1 - q^{-1}}{m_r} \frac{e^l}{1 - e^{(1 - q) l}} (1 - e^{-ql})
\end{align}
where
\begin{equation}
	l = \log \left( \frac{m_r}{x_0 H} \right)
\end{equation}

Next we turn to the time integration. For this we plug the previous result into \eqref{eq:number_density_from_F} and use the formulas for the rate $\Gamma$ and the string tension $\mu$
\begin{align}
	n_a(t) = \left( \frac{x_0}{2 m_r} \right)^{-3/2} e^{-3/2 l_\mathrm{max}} \xi \int_{-\infty}^{l_\mathrm{max}} \mathrm{d} l \xi \, \frac{e^l x_0}{2 m_r} \pi f_a^2 l  \left( \frac{x_0}{2 m_r} \right)^{-3/2} e^{-3/2 l} \frac{1 - q^{-1}}{m_r} \frac{e^l}{1 - e^{(1 - q) l}} (1 - e^{-ql})
\end{align}
Here we used 
\begin{align}
	t &= \frac{x_0 e^l}{2 m_r} \\
	H  &= \frac{e^{l} m_r}{x_0} \\
	\mathrm{d} t &= \frac{e^l x_0}{2 m_r} \mathrm{d} l \\
	R(t) &= \sqrt{t} \\
	R(t)^3 / t^3 &= 1 / t^{3/2}
\end{align}
as well as that $\xi$ is approximatly constant (we could do the computation also with a non constant $\xi$).

Finally we divide by 
\begin{equation}
	B = \frac{8 H \xi(t) \mu(t)}{x_0}
\end{equation}
to remove dimensions and the major time-dependence 
and obtain the function $f$
\begin{equation}
	f(q, l_\mathrm{max}) = n_a(t) / B(t) = \frac{1}{2} (1 - q^{-1}) \frac{e^{- \frac{1}{2} l_\mathrm{max}}}{l_\mathrm{max}} \int_{-\infty}^{l_\mathrm{max}} \mathrm{d} l \, l \frac{e^{l/2}}{1 - e^{(1 - q) l}} (1 - e^{-ql})
\end{equation}

\newpage
\printbibliography

\end{document}
