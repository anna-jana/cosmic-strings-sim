\documentclass[a4paper]{article}

\usepackage[utf8]{inputenc}
\usepackage{latexsym,amssymb,amsmath,amstext}
\usepackage{graphicx}
\usepackage[backend=bibtex, sorting=none]{biblatex}
\usepackage{subfigure}
\usepackage{float}

\bibliography{refs}

\begin{document}

\section{Background Cosmology}
Using the metric
\begin{equation}
    \mathrm{d}s^2 = -\mathrm{d}t^2 + a(t) \mathrm{d} x_i \mathrm{d} x_i
\end{equation}
and the Hubble parameter in the radiation domainted era ($\rho \sim a^{-4}$)
\begin{equation}
    H = \frac{1}{2t}
\end{equation}
$a$ is the scale factor obeying the Friedmann equation
\begin{equation}
    2 M_\mathrm{pl}^2 H^2 = \rho_\mathrm{total} = \frac{\pi^2}{30} g_* T^4
\end{equation}

Define the conformal time $\tau$ such that:
\begin{equation}
    \frac{\mathrm{d} \tau}{\mathrm{d} t} = \frac{1}{a(t)}
\end{equation}
hence:
\begin{equation}
    \frac{\mathrm{d}}{\mathrm{d} t} = \frac{\mathrm{d \tau}}{\mathrm{d} t} \frac{\mathrm{d}}{\mathrm{d} \tau} = \frac{1}{a(t)} \frac{\mathrm{d}}{\mathrm{d} \tau}
\end{equation}

\begin{align}
    \frac{\mathrm{d}^2}{\mathrm{d} t^2} = \frac{\mathrm{d}}{\mathrm{d} t} \frac{1}{a} \frac{\mathrm{d}}{\mathrm{d} \tau}
    = - \frac{\dot{a}}{a^2} \frac{\mathrm{d}}{\mathrm{d} \tau} + \frac{1}{a} \frac{\mathrm{d}}{\mathrm{d} t} \frac{\mathrm{d}}{\mathrm{d} \tau}
     = - \frac{\dot{a}}{a^2} \frac{\mathrm{d}}{\mathrm{d} \tau} + \frac{1}{a^2} \frac{\mathrm{d}^2}{\mathrm{d} \tau^2}
\end{align}

where
\begin{equation}
    \frac{\mathrm{d} 1/a}{\mathrm{d} \tau} = - a' / a^2 = - a \dot{a} / a^2 = - H
\end{equation}

\begin{equation}
    a = a_1 t^{1/2}
\end{equation}
\begin{equation}
    H = \frac{1}{2t}
\end{equation}

\begin{equation}
    \tau(t) = \int_0^t \mathrm{d} t' a^{-1}(t') = \int_0^t \mathrm{d} t' a_1 t^{-1/2} = 2 a_1 t^{1/2}
\end{equation}

\section{Model}

Based on the paper by Gorghetto et al. \cite{axions_from_strings}.
We use the string identification method by \cite{axion_dark_matter_strings_and_their_cores}.
The article \cite{improved_estimation_hiramatsu} is also important.

\begin{equation}
    \mathcal{L} = - \frac{1}{2} ( \partial \phi )^2 + V(\phi)
\end{equation}

\begin{equation}
    V(\phi) = \frac{m_r^2}{2 f_a^2}\left( |\phi|^2 - \frac{f_a^2}{2} \right)^2
\end{equation}

\begin{equation}
    \varrho = T_{00} = \underbrace{\frac{1}{2} \dot{a}^2 + \frac{1}{2} | \nabla a|^2}_{= \varrho_\mathrm{axion}}
    + \underbrace{ \frac{1}{2} \dot{r}^2 + \frac{1}{2} | \nabla r|^2 + V(r) }_{= \varrho_\mathrm{radion}}
    + \left(\frac{r^2}{2 f_a^2} + \frac{r}{f_a} \right) (\dot{a}^2 + |\nabla a|^2)
\end{equation}

Equation of Motion from the Lagragian:
\begin{equation}
    \ddot{\phi} + 3 H \dot{\phi} - \nabla^2 \phi + \phi \frac{m_r^2}{f_a^2} \left( |\phi|^2 - \frac{f_a^2}{2} \right) = 0
\end{equation}

Here the nabla is with respect to physical coordinates. They are related to comoving  coordinates by $x_\mathrm{physical} = a(t) x_\mathrm{comoving}$.
In comoving coordinates in the equation of motion the nabal is to be replaced $\nabla \to \nabla / a(t)^2$.
\begin{equation}
    (\nabla f)_i = \frac{\partial f}{\partial x_i} = \frac{\partial f}{\partial a(t) x_{\mathrm{comoving}, i}} = (\tilde{\nabla} f)_i / a^2(t)
\end{equation}

% \begin{equation}
%     \ddot{\phi} + 3 H \dot{\phi} - \nabla^2 \phi + \phi m_r^2 \left( \frac{|\phi|^2}{f_a^2} - \frac{1}{2} \right) = 0
% \end{equation}

Divide by $m_r^2$ and $f_a$:
\begin{equation}
    \frac{\mathrm{d}^2 {\phi}}{\mathrm{d} t^2 f_a m_r^2} + 3 \frac{H}{m_r} \frac{\mathrm{d} \phi}{\mathrm{d} t m_r f_a} - \frac{\nabla^2 \phi}{m_r^2 f_a} + \frac{\phi}{f_a} \left( \frac{ |\phi|^2 } { f_a^2 } - \frac{1}{2} \right) = 0
\end{equation}

Using the replacements:
\begin{align}
    t' &= m_r t \\
    H' &= H / m_r \\
    x' &= m_r x \\
    \phi' &= \phi / f_a
\end{align}

we find:
\begin{equation}
    \frac{\mathrm{d}^2 \phi'}{\mathrm{d} t'^2} + 3 H' \frac{\mathrm{d} \phi'}{\mathrm{d} t'} - \nabla'^2 \phi + \phi' \left( |\phi'|^2 - \frac{1}{2} \right) = 0
\end{equation}

We rewrite the time derivative as a derivative with respect to $\tau$.
Now use unprimed quantities and use the prime as the derivative with respect to $\tau$.
\begin{equation}
    - 1/a H \phi' + 1/a^2 \phi'' + 3 H / a \phi'
    - \nabla^2 \phi + \phi \left( |\phi|^2 - \frac{1}{2} \right) = 0
\end{equation}

\begin{equation}
    \frac{\mathrm{d} a(t)}{\mathrm{d} \tau} = \frac{\mathrm{d} t}{\mathrm{d} \tau} \dot{a} = \dot{a} / \frac{\mathrm{d} \tau}{\mathrm{d} t} = \dot{a} / (1 / a(t)) = \dot{a} a
\end{equation}

\begin{equation}
    \phi = \psi / a
\end{equation}

\begin{equation}
    \frac{\mathrm{d}}{\mathrm{d} \tau} ( \psi / a(t) ) = \frac{\mathrm{d} \psi}{\mathrm{d} \tau} / a(t) - \frac{\mathrm{d} a(t)}{\mathrm{d} \tau} \psi / a^2(t)
    = \frac{\mathrm{d} \psi}{\mathrm{d} \tau} / a(t) - \dot{a} a \psi / a^2(t)
    = \psi' / a - \psi H
\end{equation}

\begin{equation}
    \frac{d H}{d \tau} = \frac{d t}{d \tau} \frac{d H}{d t} = a (\ddot{a} / a - \dot{a}^2 / a^2) = \ddot{a} - a H^2
\end{equation}

\begin{equation}
    \frac{d^2}{d \tau^2} (\psi / a) = \frac{d}{d \tau} (\psi' / a - \psi H)
    = \psi'' / a - \psi' H - (\psi' H + \psi \frac{d H}{d \tau})
    = \psi'' / a - 2 \psi' H - \psi ( \ddot{a} - a H^2 )
\end{equation}

\begin{equation}
    1/a^2 ( \psi'' / a - 2 \psi' H - \psi ( \ddot{a} - a H^2 ) ) + 2 H / a (\psi' / a - \psi H)
    - \nabla^2 \psi / a + \psi / a \left( |\psi / a|^2 - \frac{1}{2} \right) = 0
\end{equation}

\begin{equation}
    \psi'' / a^3 - 2 \psi' H / a^2 - \psi \ddot{a} / a^2 + \psi H^2 / a
    + 2 \psi' H / a^2 - 2 H^2 \psi / a
    - \nabla^2 \psi / a + \psi / a^3 ( |\psi|^2 - \frac{a^2}{2} ) = 0
\end{equation}

\begin{equation}
    \psi'' / a^3 - \psi \ddot{a} / a^2 - H^2 \psi / a
    - \nabla^2 \psi / a + \psi / a^3 ( |\psi|^2 - \frac{a^2}{2} ) = 0
\end{equation}

\begin{equation}
    \psi'' + \psi (\underbrace{- \ddot{a} a - H^2 a^2}_{= 0})
    - \nabla^2 \psi a^2 + \psi ( |\psi|^2 - \frac{a^2}{2} ) = 0
\end{equation}

Finally we go to comoving coordinates $\tilde{x}$ defined by
\begin{equation}
    \tilde{x} = x' / a
\end{equation}
and denote the spacial derivative with respect to the comoving coordinates by $\tilde{\nabla}$.

\begin{equation}
    \psi'' - \tilde{\nabla}^2 \psi + \psi ( |\psi|^2 - \frac{a^2}{2} ) = 0
\end{equation}

\begin{equation}
    \phi_\mathrm{physical} = f_a \psi / a(t)
\end{equation}

\section{Energies}

\subsection{Decomposition into Axion and Radial Mode}
\begin{equation}
    \phi(x) = \frac{r(x) + f_a}{\sqrt{2}} e^{i \varphi(x) / f_a }
\end{equation}

\subsection{The Hamiltonian Density}
In physical coordinates and field basis:
\begin{equation}
    \mathcal{H} =
    \underbrace{\frac{1}{2} \dot{\varphi}^2 + \frac{1}{2} (\nabla \varphi)^2}_{= \mathrm{axion}} +
    \underbrace{\frac{1}{2} \dot{r}^2 + \frac{1}{2} (\nabla r)^2 + V(r)}_{= \mathrm{radial}} +
    \underbrace{\left( \frac{r^2}{f_a^2} + 2 \frac{r}{f_a} \right) \left( \frac{1}{2} \dot{\varphi}^2 + \frac{1}{2} (\nabla \varphi)^2 \right)}_{= \mathrm{interaction}}
\end{equation}

\subsection{Axion:}

field definition and conversion:
\begin{equation}
    \varphi = f_a \theta = f_a \arg(\phi_\mathrm{physical}) = f_a \arg(\psi)
\end{equation}

kinetic:
\begin{align}
    &\frac{1}{2} \dot{\varphi}^2 = \frac{1}{2} f_a^2 \dot{\theta}^2
    = \frac{1}{2} f_a^2 \left( \frac{\mathrm{d} \theta}{\mathrm{d} t} \right)^2
    = \frac{1}{2} f_a^2 \left( \frac{\mathrm{d} \theta}{\mathrm{d} t' / m_r} \right)^2 \\
    &= \frac{1}{2} f_a^2 m_r^2 \left( \frac{\mathrm{d} \tau}{\mathrm{d} t'} \frac{\mathrm{d} \theta}{\mathrm{d} \tau} \right)^2
    = \frac{1}{2} f_a^2 m_r^2 \left( \frac{1}{a(t)} \frac{\mathrm{d} \theta}{\mathrm{d} \tau} \right)^2
    = \frac{1}{2} \left( \frac{f_a m_r}{a(t)} \right)^2 \left( \partial_\tau \theta \right)^2
\end{align}

\begin{align}
    \partial_\tau \theta &= \partial_\tau \arg(\phi) = \partial_\tau \arg(\psi) = \partial_\tau \arctan \left( \frac{ \Im \psi }{ \Re \psi } \right)
    = \frac{ \frac{ \partial_\tau \Im \psi }{ \Re \psi } - \frac{ \Im \psi \cdot \partial_\tau \Re \psi}{(\Re \psi)^2}}{ 1 + \left( \frac{\Im \psi}{\Re \psi} \right)^2} \\
                         &= \frac{ \Re \psi \partial_\tau \Im \psi - \Im \psi \cdot \partial_\tau \Re \psi}{ (\Re \psi)^2 + (\Im \psi)^2} =  \frac{ \Re \psi \partial_\tau \Im \psi - \Im \psi \cdot \partial_\tau \Re \psi}{|\psi|^2}
\end{align}

gradient:
\begin{align}
 &\frac{1}{2} (\nabla \varphi)^2 =
 \frac{1}{2} f_a^2 \sum_i \left(\frac{\partial \theta}{\partial x_i}\right)^2 = \\
 &\frac{1}{2} f_a^2 \sum_i \left(\frac{\partial \theta}{\partial x'_i / m_r}\right)^2 =
 \frac{1}{2} f_a^2 m_r^2 \sum_i \left(\frac{\partial \theta}{\partial x'_i}\right)^2 =
 \frac{1}{2} f_a^2 m_r^2 (\nabla' \theta)^2
\end{align}

\begin{equation}
    \frac{1}{2} (\nabla \varphi)^2 = \frac{1}{2} f_a^2 m_r^2 (\tilde{\nabla} \varphi)^2 / a(t)^2
\end{equation}

\subsection{Radial Mode:}

Field definition and conversion
\begin{equation}
    \phi(x) = \frac{r(x) + f_a}{\sqrt{2}} e^{i \theta(x) }
\end{equation}

\begin{equation}
    |\phi| = \frac{r + f_a}{\sqrt{2}}
\end{equation}

\begin{equation}
    r = \sqrt{2} |\phi| - f_a
    = \sqrt{2} f_a |\psi| / a(t) - f_a
    = f_a \left( \sqrt{2} |\psi| / a(t) - 1 \right)
    \equiv f_a r'
\end{equation}

\begin{equation}
    r' = r / f_a = \sqrt{2} |\psi| / a(t) - 1
\end{equation}

kinetic:
\begin{equation}
    \frac{1}{2} \dot{r}^2 = \frac{1}{2} (\frac{\partial}{\partial t} r)^2
    = \frac{1}{2} (\frac{f_a m_r}{a(t)})^2 \left( \partial_\tau r' \right)^2
\end{equation}

\begin{align}
    \partial_\tau r' = \partial_\tau \left( \sqrt{2} |\psi| / a - 1 \right)
    &= \sqrt{2} \partial_\tau ( |\psi| / a ) \\
    &= \sqrt{2} (\partial_\tau |\psi| / a - \partial_\tau a |\psi| / a^2) \\
    &= \sqrt{2} (\partial_\tau |\psi| / a - \dot{a} a |\psi| / a^2) \\
    &= \sqrt{2} (\partial_\tau |\psi| / a - H |\psi|)
\end{align}

\begin{align}
    \partial_\tau |\psi| &= \partial_\tau \sqrt{ (\Re \psi)^2 + (\Im \psi)^2 } \\
                         &= \frac{ \partial_\tau ((\Re \psi)^2 + (\Im \psi)^2) }{ 2 \sqrt{ (\Re \psi)^2 + (\Im \psi)^2 } } \\
                         &= \frac{ 2 \Re \psi \partial_\tau \Re \psi + 2 \Im \psi \partial_\tau \Im \psi }{ 2 \sqrt{ (\Re \psi)^2 + (\Im \psi)^2 } } \\
                         &= \frac{ \Re \psi \partial_\tau \Re \psi + \Im \psi \partial_\tau \Im \psi }{ |\psi| }
\end{align}

gradient:
\begin{equation}
    \frac{1}{2} (\nabla r)^2 = \frac{1}{2} f_a^2 m_r^2 ( \nabla' r')^2
    = \frac{1}{2} f_a^2 m_r^2 ( \tilde{\nabla} r')^2 / a(t)^2
\end{equation}

potential:
\begin{equation}
    V(\phi) = \frac{m_r^2}{2 f_a^2}\left( |\phi|^2 - \frac{f_a^2}{2} \right)^2
\end{equation}

\begin{align}
    &V(\phi) = V(|\phi|) = V(\frac{r + f_a}{\sqrt{2}})
    = V \left(f_a \left( \frac{r' + 1}{\sqrt{2}} \right) \right) \\
    &= \frac{m_r^2}{2 f_a^2}\left( \left(f_a \left( \frac{r' + 1}{\sqrt{2}} \right) \right)^2 - \frac{f_a^2}{2} \right)^2 \\
    &= m_r^2 f_a^2 / 8 ( (r' + 1)^2 - 1)
    = m_r^2 f_a^2 / 8 ( r'^2 + 2 r' )^2
\end{align}

\subsection{Interaction:}
\begin{equation}
\left( \frac{r^2}{f_a^2} + 2 \frac{r}{f_a} \right) \left( \frac{1}{2} \dot{\varphi}^2 + \frac{1}{2} (\nabla \varphi)^2 \right)
= f_a^2 m_r^2 (r'^2 + 2 r') \left( \frac{1}{2} (\partial_\tau \theta )^2 + \frac{1}{2} (\tilde{\nabla} \theta)^2 / a(t)^2 \right)
\end{equation}

\section{Spectrum Estimation}

We use the method described in \cite[sec. 3.2]{improved_estimation_hiramatsu}.

\begin{equation}
    \frac{1}{2} \langle \dot{a}(\vec{k}, t)^* \dot{a}(\vec{k}', t) \rangle = \frac{(2\pi)^3}{k^2} \delta^{(3)}(\vec{k} - \vec{k}') P(k, t)
\end{equation}

\begin{equation}
    \tilde{\dot{a}}(\vec{x}_\mathrm{phys}) \equiv W(\vec{x}_\mathrm{phys}) \dot{a}(\vec{x}_\mathrm{phys})
\end{equation}

\begin{equation}
    \tilde{P}(k) \equiv \frac{k^2}{V} \int \frac{d \Omega_k}{4\pi} \frac{1}{2} \left| \tilde{\dot{a}}( \vec{k} ) \right|^2
\end{equation}

\begin{equation}
    \hat{P}(k) \equiv \frac{k^2}{V} \int \frac{d k'}{2 \pi^2} M^{-1}(k, k') \tilde{P}(k')
\end{equation}

\begin{equation}
    \int \frac{k'^2 d k'}{2 \pi^2} M^{-1}(k, k') M(k', k'') = \frac{2\pi^2}{k^2} \delta(k - k'')
\end{equation}

\begin{equation}
    M(k, k') \equiv \frac{1}{V^2} \int \frac{d \Omega_k}{4 \pi} \frac{d \Omega_{k'}}{4 \pi} \left| W(\vec{k} - \vec{k'}) \right|^2
\end{equation}

\begin{enumerate}
    \item Detect Strings
    \item Compute $W(\vec{x})$
    \item Compute $\dot{a}(\vec{x})$
    \item Compute $\tilde{\dot{a}}(\vec{x}) = a(\vec{x}) \cdot W(\vec{x})$
    \item Compute $\tilde{P}(k)$ from $\tilde{\dot{a}}(\vec{x})$
    \item Compute FFT of $W(\vec{x})$
    \item Compute $M(k, k')$ from $W(\vec{k})$
    \item Invert $M(k, k')$
    \item Compute $\hat{P}(k)$ using $M^{-1}(k, k')$ and $\tilde{P}(k)$
\end{enumerate}

\section{Instantaneous Emission Spectrum}
We define the instantaeous emission spectrum into axions as
\begin{equation}
    F[x = k / H, y = m_r / H] = A / a^3 \frac{\partial}{\partial t} \left( a^3 \frac{\partial \rho_a}{\partial k} \right)
\end{equation}

\section{String Length}

We define the string length per Hubble volume as
\begin{equation}
    \zeta \equiv \lim_{V \to \infty} \frac{l}{vol(V)} t^2
\end{equation}
where $l$ is the total length of strings within $V$.


\newpage
\printbibliography

\end{document}
